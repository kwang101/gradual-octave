%%%% Proceedings format for most of ACM conferences (with the exceptions listed below) and all ICPS volumes.
%\documentclass[sigconf]{acmart}
%%%% As of March 2017, [siggraph] is no longer used. Please use sigconf (above) for SIGGRAPH conferences.

%%%% Proceedings format for SIGPLAN conferences 
\documentclass[sigplan, xcolor=usenames,dvipsnames]{acmart}

%%%% Proceedings format for SIGCHI conferences
% \documentclass[sigchi, review]{acmart}

%%%% To use the SIGCHI extended abstract template, please visit
% https://www.overleaf.com/read/zzzfqvkmrfzn


\usepackage{booktabs} % For formal tables
\usepackage{mathpartir} % available from http://cristal.inria.fr/~remy/latex/

\usepackage{listings}
\usepackage{soul}

\lstset{% general command to set parameter(s) 
basicstyle=\ttfamily\footnotesize, % print whole listing small 
keywordstyle=\bfseries,% bold black keywords 
identifierstyle=, % nothing happens
numbers=left,                   % where to put the line-numbers
%stringstyle=\ttfamily, % typewriter type for strings
stepnumber=1,                   % the step between two line-numbers. If it is 1 each line will be numbered
numbersep=3pt,                  % how far the line-numbers are from the code
numberstyle=\tiny\color{gray},
%xleftmargin=5.0ex,
escapechar=!,
commentstyle=\itshape\color{OliveGreen},
%escapeinside={<@}{@>},
showstringspaces=false % no special string spaces, because fugly.
}

\definecolor{light-gray}{gray}{0.40}

% must look at <https://tex.stackexchange.com/questions/8851/how-can-i-highlight-some-lines-from-source-code>
% for fancy boxes/highlighting in lstlistings

\lstdefinelanguage{racket} {
  morekeywords=[1]{define, define-syntax, define-macro, lambda, define-stream, stream-lambda},
  morekeywords=[2]{begin, call-with-current-continuation, call/cc,
    call-with-input-file, call-with-output-file, case, cond,
    do, else, for-each, if,
    let*, let, let-syntax, letrec, letrec-syntax,
    define-context, define-controller, Integer, Boolean, get, when-required, when-provided,
    maybe_publish, require, submod, or/c, ->, \#\%module-begin,
    always_publish, with-syntax, define-struct/contract, syntax-case,
    define/contract,
    let-values, let*-values,
    module, provide,
    and, or, not, delay, force,
    \#`, \#',
    \#lang, implement, begin-for-syntax, rename-out,
    quasiquote, quote, unquote, unquote-splicing,
    map, fold, syntax, syntax-rules, eval, environment, query },
  morekeywords=[3]{import, export},
  alsoletter={',`,-,/,>,<,\#,\%},
  morecomment=[l]{;},
%  literate={lambda}{{\lambdaup}}1, % lambda -- look at https://tex.stackexchange.com/questions/119879/math-symbols-in-tt-font
  moredelim=**[is][\color{light-gray}]{<<@<<}{>>@>>},
  moredelim=**[is][\itshape\color{OliveGreen}]{<<;<<}{>>;>>},
  morecomment=[s]{\#|}{|\#},
  sensitive=true,
}

\lstdefinestyle{box}{
  keywordstyle=\color{blue},
  commentstyle=\color{green},
}

\def\inlineracket{\lstinline[basicstyle=\ttfamily\normalsize,language=racket]}

%%% Local Variables: 
%%% mode: latex
%%% TeX-master: "progfw.main"
%%% End: 


% Copyright
% \setcopyright{none}
%\setcopyright{acmcopyright}
%\setcopyright{acmlicensed}
\setcopyright{rightsretained}
%\setcopyright{usgov}
%\setcopyright{usgovmixed}
%\setcopyright{cagov}
%\setcopyright{cagovmixed}
\settopmatter{printacmref=false}
\renewcommand\footnotetextcopyrightpermission[1]{}
\pagestyle{plain}


\begin{document}
\title{Plan/Proof-of-Concept: \\Gradual Typing for Octave Language}
\titlenote{Permission to make digital or hard copies of part or all of this work
for personal or classroom use is granted without fee provided that copies are
not made or distributed for profit or commercial advantage and that copies bear
this notice and the full citation on the first page. Copyrights for third-party
components of this work must be honored. For all other uses, contact the
owner/author(s).}
\subtitle{University of British Columbia CPSC 311 Course Project}


\author{Ada Li}
\affiliation{%
  \institution{University of British Columbia}
}
\email{adali@alumni.ubc.ca}

\author{Yuchong Pan}
\affiliation{%
  \institution{University of British Columbia}
}
\email{yuchong.pan@alumni.ubc.ca}

\author{Kathy Wang}
\affiliation{%
  \institution{University of British Columbia}
}
\email{kathy.wang@alumni.ubc.ca}

\author{Paul Wang}
\affiliation{%
  \institution{University of British Columbia}
}
\email{paul.wang@alumni.ubc.ca}


\begin{abstract}
In this background report for our proposed project, we explore advantages of applying gradual typing for Octave and additionally discuss the proposed project in the context of existing work. We contend the novelty of the project in the context of existing work in the domain and also provide an in-depth tutorial on a proof-of-concept we have constructed to demonstrate a potential approach. Finally, we provide a plan and overview of the remaining work for the project.
\end{abstract}


\keywords{gradual typing, Octave}


\maketitle

\section{Introduction}

\appendix

\begin{acks}
    The authors would like to thank the course staff of CPSC 311 at University
    of British Columbia for their feedback.
\end{acks}


\bibliographystyle{ACM-Reference-Format}
\bibliography{project-plan-proof-bibliography}

\end{document}
