\section{Introduction}
Static and dynamic type systems have their respective advantages. Static typing
allows early error detection and enforces code style in a collaborative setting.
It is, however, acknowledged that dynamic languages are better for fast
prototyping. Over the past several decades, researchers in the programming
language community have been working on integrating static typing and dynamic
typing in a single language, having programmers control the level of type
annotations. Gradual typing is a solution to combine the two type systems,
proposed by \cite{siek2006gradual}. It is of increasing interest in the
programming language community and has been adopted by many programming
languages, both in the industry and in the academia, such as Typed Racket
\cite{tobin2006interlanguage}, TypeScript \cite{bierman2014understanding} and
Reticulated Python \cite{vitousek2014design}.

Octave is a scientific programming language with a dynamic type system. Because
it has powerful matrix operations and is compatible with MATLAB scripts, it is
widely used in statistics, mathematics and computer science communities for idea
validation and fast prototyping. In this project, we propose a gradual type
system for the Octave language to allow Octave programmers to annotate source
code with optional type annotations, making Octave programs more robust and more
suitable for production environments.

\section{Future Work}
Because of the time limitation of the course, this project will be mainly
focusing on the static semantics of the proposed gradual type system for Octave.
In the future, however, more work can be done to extend the proposed gradual
type system. For instance, dynamic type checking could be added during the
interpretation, properly tracking errors from source code. This can be done by
translation into an internal cast calculus with the blame tracking
\cite{siek2015refined}. In addition to that, type inference could be added to
automatically deduce types at compile time. \cite{garcia2015principal}
introduces an approach to gradual type inference.

\section{Related Work}
There have been many research publications and projects on adding gradual type
systems to an existing dynamic language. Diamondback Ruby (DRuby) is an
extension to the Ruby language that combines static and dynamic typing in Ruby
with constraint-based type inference \cite{furr2009combining}. Typed Racket
\cite{tobin2006interlanguage} recently began to support refinement and dependent
function types as experimental features \cite{racket2017refinement}. Other
similar projects include Typed Lua \cite{maidl2014typed} and Gradualtalk
\cite{allende2014gradual} that brings an optional type system to Lua and to
Smalltalk, respectively. Our project would largely benifit from these projects.

\appendix

\begin{acks}
    The authors would like to thank the course staff of CPSC 311 at University
    of British Columbia for their feedback.
\end{acks}
