\section{Introduction}
Static and dynamic type systems have their respective advantages. Static typing
allows early error detection and enforces code style in a collaborative setting.
It is, however, acknowledged that dynamic languages are better for fast
prototyping. Over the past several decades, researchers in the programming
language community have been working on integrating static typing and dynamic
typing in a single language, having programmers control the level of type
annotations. Gradual typing is a solution to combine the two type systems,
proposed by \cite{siek2006gradual}. It is of increasing interest in the
programming language community and has been adopted by many programming
languages, both in the industry and in the academia, such as Typed Racket
\cite{tobin2006interlanguage}, TypeScript \cite{bierman2014understanding} and
Reticulated Python \cite{vitousek2014design}.

Octave is a scientific programming language with a dynamic type system. Because
it has powerful matrix operations and is compatible with MATLAB scripts, it is
widely used in statistics, mathematics and computer science communities for idea
validation and fast prototyping. In this project, we propose a gradual type
system for the Octave language to allow Octave programmers to annotate source
code with optional type annotations, making Octave programs more robust and more
suitable for production environments.

\section{Overview and Plans}
In this project, we will add a gradual type system to the Octave language.
Because of the time limitation of the course, we will be focusing on the static
semantics. Future work is discussed in Section \ref{sec-future-work}. This means
that our proposed gradual type system will mainly be used for static analysis
and IDE tooling (e.g., code completion and refactoring, etc.), but not for
runtime type checking. Therefore, this project is similar to type hints in
Python \footnote{New in version 3.5.} and the type system of TypeScript.

\subsection{How to Fulfill Background Research Report Milestone}
We will be using the resources listed in Section \ref{sec-supporting-resources}
as starting points and guidelines for us to work from as we explore the articles
that focuses on gradual typing. We will cover its value in everyday programming
practice as well as cover the technical elements.

\subsection{How to Fulfill Proof-of-Concept Milestone}
For the proof-of-concept milestone, we will be focusing on implementing a static
type checker for the gradually-typed lambda calculus (GTLC) by following
\cite{siek2012interpretations}. The gradually-typed lambda calculus can be
regarded a core of gradually typed programming languages, and implementing a
static type checker for it will help us gain deeper understanding of gradual
type systems. The following grammar defines the syntax of the gradually-typed
lambda calculus \cite{siek2012interpretations}.

$$ \text{variables} \ x \quad \text{integers} \ n \quad \text{blame labels} \ l $$
$$
\begin{array}{llll}
    \text{basic types} & B & ::= & \text{\tt int} \ | \ \text{\tt bool} \\
    \text{types} & T & ::= & B \ | \ (\rightarrow T \ T) \ | \ \text{?} \\
    \text{constants} & k & ::= & n \ | \ \text{\tt \# t} \ | \ \text{\tt \# t} \\
    \text{operators} & op & ::= & \text{\tt inc} \ | \ \text{\tt dec} \ | \ \text{\tt zero?} \\
    \text{expressions} & e & ::= & k \ | \ (op \ e \ l) \ | \ (\text{\tt if} \ e \ e \ e \ l) \ | \ x \ | \ (e \ e \ l) \ | \ \\
    & & & (\lambda (x) \ e) \ | \ (\lambda (x : T) \ e) \ | \ (e : T \ l)
\end{array}
$$

In addition, we are going to experiment various approaches to enforce type
checking for matrix operations (e.g., matrix dimension checking), which are a
significant component of the Octave language.

\subsection{How to Fulfill Final Project Milestone}
For the final project milestone, we would like to apply what we will have
learned in the gradually-typed lambda calculus to the Octave language. To
achieve this, we plan to implement a parser that converts Octave source code to
abstract syntax trees in Racket, using parsing tools available in Racket.
For practical reasons, we are going to implement the most significant subset of
the Octave language to demonstrate the features of Octave. After parsing, we
will implement a static type checker with matrix operation support for abstract
syntax trees.

\subsection{How to Fulfill Poster Milestone}
Our poster will emphasize the usefulness it has for users when dealing with
matrices, which will entice our peers to learn more about the project as well
possibly integrating it. This is particularly useful for students involved in
mathematics and statistic courses as they will be using software like Octave
frequently. We will be including diagrams on how this project works so students
can easily understand without having to read a large amount of text. We will try
to include an easy way for students to learn more about this project topic. Some
ideas include compiling a list of topics on a single webpage and including that
on our poster. Another idea would be creating a QR code for students to scan and
receive more information.

\subsection{How This Can Be a Low-Risk Approach}
This project has a low risk because researchers in the gradual typing community
have been exploring designs of gradual type systems for many years. There have
been many publications in this field and other related fields, among which we
have compiled a list of resources that may be of particular importance for our
project, listed in Section \ref{sec-supporting-resources}. For the language choice,
Octave is an open source language compatible with MATLAB. This avoids copyright
issues and ensures the audience of our project. In addition, there are many open
source projects related to MATLAB/Octave parsing, which will help us with the
parsing of Octave source code.

\section{Future Work}
\label{sec-future-work}
Because of the time limitation of the course, this project will be mainly
focusing on the static semantics of the proposed gradual type system for Octave.
In the future, however, more work can be done to extend the proposed gradual
type system. For instance, dynamic type checking could be added during the
interpretation, properly tracking errors from source code. This can be done by
translation into an internal cast calculus with the blame tracking
\cite{siek2015refined}. In addition to that, type inference could be added to
automatically deduce types at compile time. \cite{garcia2015principal}
introduces an approach to gradual type inference.

\section{Related Work}
There have been many research publications and projects on adding gradual type
systems to an existing dynamic language. Diamondback Ruby (DRuby) is an
extension to the Ruby language that combines static and dynamic typing in Ruby
with constraint-based type inference \cite{furr2009combining}. Typed Racket
\cite{tobin2006interlanguage} recently began to support refinement and dependent
function types as experimental features \cite{racket2017refinement}. Other
similar projects include Typed Lua \cite{maidl2014typed} and Gradualtalk
\cite{allende2014gradual} which bring optional type systems to Lua and to
Smalltalk, respectively. Our project would largely benifit from these projects.

\section{Supporting Resources}
\label{sec-supporting-resources}
We have compiled a list of resources that may be of particular importance for
our project, listed as follows.

\begin{itemize}
    \item \url{https://dl.acm.org/citation.cfm?id=2661112}: This will be one of
        our main supporting documents. This paper serves as a tutorial on how to
        type-check and interpret the gradually-typed lambda calculus. We will
        use this as a guide when we implement our proposed gradual typing
        system.
    \item \url{http://homes.sice.indiana.edu/mvitouse/papers/snapl15.pdf}: This
        paper provides a formal characterization of behaviors of gradual type
        systems. It also surveys popular programming languages with gradual type
        systems. There are also very useful examples in the paper which we can
        reference.
    \item \url{http://www.cs.umd.edu/projects/PL/druby/}: This project adds type
        annotations, type inference and dynamic type checking to Ruby. We can
        use this project as a reference when we design the proposed gradual type
        system for Octave.
    \item \url{https://github.com/samth/gradual-typing-bib}: This link contains
        a list of publications in the field of gradual typing. In case the links
        above do not provide enough information for our project, we will still
        have a large pool of resources to look through.
    \item \url{https://github.com/ewiger/decade/tree/master/lib/mparser} and
        \url{https://github.com/ericharley/matlab-parser}: These two projects
        are both MATLAB parsers that we can reference. Since the syntax of
        Octave is very similar to MATLAB, we can get valuable insight from these
        parsers.
\end{itemize}

\section{Summary}
The objective for this project is to develop a gradually-typed variant for
Octave that sufficiently expresses the basic data types of the language. We
believe that the flexible nature of gradual typing offers great value to data
scientists and various users of Octave in that it remains an effective
prototyping tool while optionally providing compile-time type and invariant
assertions. To achieve this, we intend to base our initial prototypes for static
semantics on the existing literature described above. Additionally -- and as
time permits -- we intend to enrich and further introduce nuances into our
gradual typing scheme by delving into features such as static dimension checking
(as matrices are a core component of the language domain) and other more
domain-specific aspects. We hope to create a meaningful and useful tool for
Octave developers to use and ideally contribute towards in the future.

\appendix

\begin{acks}
    The authors would like to thank the course staff of CPSC 311 at University
    of British Columbia for their feedback.
\end{acks}
